\documentclass[12pt,a4paper]{scrartcl}
\usepackage[utf8]{inputenc}
\usepackage[english,russian]{babel}
\usepackage{indentfirst}
\usepackage{misccorr}
\usepackage{graphicx}
\usepackage{amsmath}
\begin{document}
\begin{titlepage}
\begin{center}
\large

Интерполяция многочленом.

\end{center}
\vfill
\begin{flushright}
Лукьянчиков Иван

Группа 424
\end{flushright}
\end{titlepage}
\textbf{1) Постановка задачи.}
Имеется функция $f$. Заданны на узлы $x_0,x_1, \dots x_N$ и известно значение $f$ в этих узлах $f_0,_f_1, \dots f_N$. Требуется построить полином $N$ степени, проходящий через $N+1$ точку $(x_i,f_i)$.

\textbf{2) Решение задачи.}
Искомы полином может быть получен следующими способами:
1)По формуле интерполяционного многочлена Лагранжа
$$P_n^*(x)=\sum_{i=0}^n y_i\prod_{j=0,j \neq i}^n \dfrac{x-x_j}{x_i-x_j}$$
2)Решив систему

\begin{center}

$\begin{pmatrix}
f_{0} \\
f_{1}\\
\vdots \\
f_{N}
\end{pmatrix} = c_0*\begin{pmatrix}
x_{0}^N \\
x_{1}^N\\
\vdots \\
x_{N}^N
\end{pmatrix} + \dots + c_N*\begin{pmatrix}
x_{N}^0 \\
x_{N}^0\\
\vdots \\
x_{N}^0
\end{pmatrix}$
\end{center}

\textbf{3) Реализация задачи.}
Задача была реализована на языке программирования С++. Мною был разработан class myvector и в рамках этого класса были реализованны следующие функции: сложение векторов, умножение вектора на скаляр, и скалярное произведение векторов. Также была реализованна функция решающая систему линейных уравнений методом Гаусса. И все посчитано.

\textbf{4) Тестирование.}
На отрезке [-1,1] была взята равномерная сетка из 9 узлов и значения функции $10*x^8+x^3+3x^2+7$, и по данным значениям строился полином 8 степени. Были посчитаны коэффициенты для искомого полинома и на равномерной сетке из 17 узлов рассмотрено отклонение от искомой функции. Также на равномерной сетке из 17 рассмотрено отклонение от искомой функции значений полученных по формуле интерполяционного многочлена Лагранжа.

1)

delta_g=+0.39813

2)

delta_l=+8.67362e-19

Аналогичная процедура была проделана на отрезке [-1,1] для $|x|$, который приближался полиномом 12 степени.

1)

delta_g=+97.8128

2)

delta_l=+1.20607

\end{document}
